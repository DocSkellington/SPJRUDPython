\documentclass[]{article}

\usepackage{qtree}

\usepackage{listings}

\usepackage{hyperref}

\usepackage{amsthm}
\usepackage{amsmath}

\usepackage[frenchb]{babel}

\newtheorem{SPJRUDRequest}{Requ\^ete SPJRUD}
\newtheorem{request}{Requ\^ete}

\title{Rapport du projet de Base de Donn\'ees I}
\author{DE COOMAN Thibaut \and STAQUET Ga\"etan}

\begin{document}
\maketitle

\tableofcontents

\section{Introduction}
Comme conseill\'e dans l'\'enonc\'e, nous avons employ\'e le principe de l'AST pour impl\'ementer ce v\'erificateur et traducteur.

Dans ce rapport, nous parlerons de l'utilisation de l'application, de la syntaxe \`a employer pour rentrer les commandes, de la fa\c{c}on dont l'AST est construit, de sa v\'erification et de sa traduction en SQL, ainsi que de la communication avec la base de donn\'ees.

Nous supposons que nous travaillons dans la base de donn\'ees suivante:

\begin{center}
	\begin{tabular}{| c | c | c |}
		\hline
		\textbf{Name} & \textbf{Country} & \textbf{Population} \\
		\hline
		Bruxelles & Belgique & 184230\\
		\hline
		Paris & France & 123456789\\
		\hline
	\end{tabular}
	\label{Cities}
\end{center}

\section{Utilisation}
L'application permet de travailler sur une base de donn\'ees d\'ej\`a existante ou sur un sch\'ema que l'utilisateur doit entrer. Une fois ceci fait, l'utilisateur peut entrer une requ\^ete SPJRUD. L'application v\'erifie cette requ\^ete et la traduit en SQL. Si l'utilisateur a demand\'e \`a employer une base de donn\'ees, l'application ex\'ecute (via SQLite) la requ\^ete SQL et affiche le r\'esultat. Dans tous les cas, l'application demande \`a l'utilisateur d'entrer une nouvelle requ\^ete (ou une ligne vide pour quitter l'application).

\section{Syntaxe}
\subsection{SPJRUD}
Les requ\^etes SPJRUD doivent respecter la syntaxe d\'ecrite dans la table suivante. $E$ indique une relation alg\'ebrique et $E'$ indique cette relation \'ecrite dans la syntaxe demand\'ee par l'application.

\begin{center}
	\begin{tabular}{| c | c |}
		\hline
		\textbf{SPJRUD} & \textbf{Application}\\
		\hline
		Relation R & Rel("R")\\
		\hline
		$\sigma_{A='a'}(E)$ & Select(Eq("A", Cst('a')), $E'$)\\
		\hline
		$\sigma_{A=B}(E)$ & Select(Eq("A", Col("B")), $E'$)\\
		\hline
		$\pi_{X}(E)$ & Proj(["$X_1$", "$X_2$", "$X_3$", \ldots, "$X_n$"], $E'$)\\
		\hline
		$E_1 \bowtie E_2$ & Join($E_1'$, $E_2'$)\\
		\hline
		$\rho_{A \to C}(E)$ & Rename("A", "C", $E'$)\\
		\hline
		$E_1 \cup E_2$ & Union($E_1'$, $E_2'$)\\
		\hline
		$E_1 - E_2$ & Diff($E_1'$, $E_2'$)\\
		\hline
	\end{tabular}
\end{center}

\subsection{Sch\'emas}
La syntaxe \`a employer pour d\'efinir des sch\'emas de relation est la suivante:

\begin{center}
	\begin{tabular}{p{250px}}
		"Nom de la relation", ("Nom de la colonne 1", "Type de la colonne (SQL types)", "Si la colonne peut contenir la valeur NULL ou non"), ("Nom de la colonne 2", \ldots), \ldots
	\end{tabular}
\end{center}

Par exemple,
\begin{center}
	\begin{tabular}{p{250px}}
		"Notes", ("Nom", "VARCHAR(25)", False), ("Points", "INTEGER", False)
	\end{tabular}
\end{center}

d\'efinirait la table suivante:
\begin{center}
	\begin{tabular}{| c | c |}
		\hline
		\textbf{Nom} & \textbf{Points}\\
		\hline
	\end{tabular}
\end{center}

Nous n'avons pas permis de populer une table d\'efinie de cette fa\c{c}on.

\section{Construction de l'AST}

Admettons que l'utilisateur veuille traduire la requ\^ete en SPJRUD suivante:

\begin{SPJRUDRequest}
$\rho_{Name \to City}(\pi_{Name}(\sigma_{Country='France'}(Cities) \text{ } \cup \text{ } \sigma_{Country='Belgique'}(Cities)))$
\end{SPJRUDRequest}

Cette requ\^ete devrait \^etre encod\'ee comme:

\begin{request}\label{request1}
Rename("Name", "City", Proj(["Name"], Union(Select(Eq("Country", Cst("France")), Rel("Cities")), Select(Eq("Country", Cst("Belgique")), Rel("Cities")))))
\end{request}

Pour pouvoir construire l'arbre correspondant \`a cette requ\^ete, nous avons d\'ecid\'e de proc\'eder comme suit:

\begin{enumerate}
	\item V\'erifier les parenth\`eses et crochets
	\item D\'ecomposer la requ\^ete
	\item Construire l'arbre n\œ{}ud par nœ{}ud
\end{enumerate}

La v\'erification des parenth\`eses et crochets est suffisament simple pour ne pas \^etre expliqu\'ee ici.

\subsection{D\'ecomposition}
Cette \'etape cr\'ee une liste de listes et/ou de chaînes de caract\`eres. Par exemple, \textit{Select(Eq("A", Cst(a)), Rel("R"))} donne la d\'ecomposition \textit{["Select", ["Eq", ["A", "Cst", ["a"]], "Rel", ["R"]]]}

La requ\^ete \ref{request1} donnerait ainsi la d\'ecomposition:

\begin{center}
	\begin{tabular}{p{250px}}
		["Rename", ["Name", "City", "Proj", [["Name"], "Union", ["Select", ["Eq", ["Country", "Cst", ["France"]], "Rel", ["Cities"]], "Select", ["Eq", ["Country", "Cst", ["Belgique"]], "Rel", ["Cities"]]]]]]
	\end{tabular}\label{"decomposition"}
\end{center}

\subsection{Construction}

Cette d\'ecomposition permet de construire un algorithme r\'ecursif pour construire l'arbre. Appelons-le \textit{build_AST}. Cet algorithme prend en param\`etre la sous-liste qui doit \^etre trait\'ee (le cas de base \'etant la Relation). Ainsi, la d\'ecomposition \textit{["Select", ["Eq", ["A", "Cst", ["a"]], "Relation", ["R"]]]} donnerait l'ex\'ecution suivante:

\begin{enumerate}
    \item build_AST(["Select", ["Eq", ["A", "Cst", ["a"]], "Relation", ["R"]]])
        \begin{enumerate}
        \item build_AST(["Relation", ["R"]])
        \end{enumerate}
\end{enumerate}

Lorque \textit{build_AST} rencontre un nom d'op\'eration, il r\'ecup\`ere les valeurs n\'ecessaires \`a cette op\'eration. Par exemple, l'Union demande deux sous-requ\^etes tandis que la Relation demande le nom de la relation.

De la liste \ref{decomposition}, nous pouvons construire l'arbre suivant:

\Tree [.Rename [.Projection [.Union [.Selection [.Relation ] ] [.Selecttion [.Relation ] ] ] ] ]

Le nom indiqu\'e dans les n\œ{}uds est le nom de la classe employ\'ee.

\section{V\'erification de l'AST}
Une fois l'arbre construit, nous pouvons commencer \`a v\'erifier s'il est correct. Par \textit{correct}, nous entendons qu'il r\'epond aux exigences et \`a la logique de l'alg\`ebre relationnel (que les sch\'emas soient respect\'es, \ldots). Pour ce faire, chaque classe d\'efinit une fonction \textit{check}. Si l'op\'eration n'est pas une Relation, cette m\'ethode appelle le \textit{check} du/des n\œ{}ud(s) enfant(s). Cette m\'ethode construit \'egalement le sch\'ema qui d\'ecoule de celui du/des enfant(s) et de l'op\'eration (par exemple, la s\'election ne modifie pas le \textit{sorte} mais la projection peut retirer des colonnes).

Si une erreur est d\'etect\'ee, une Exception d\'ecrivant le probl\`eme est lanc\'ee.

\section{Traduction de l'AST}
\end{document}
